\documentclass{sig-alternate}

\usepackage{graphicx,latexsym,epsf,epsfig,epstopdf}
\usepackage{amssymb,amsmath}
\usepackage{alltt}
\usepackage{enumitem}
\usepackage{amsfonts}
\usepackage{pifont}
\usepackage{algorithm, algpseudocode}
\usepackage{float}
\usepackage{subfigure}
\usepackage[hyphens]{url}

\hbadness=10000

\usepackage{color}
\usepackage{xspace}

\newcommand{\hilight}[1]{\colorbox{yellow}{#1}}
\newcommand{\reminder}[1]{[[ \marginpar[$=>$]{\flushleft$<=$}{\bf #1} ]]}

\newcommand{\cmark}{\ding{51}}%
\newcommand{\xmark}{\ding{55}}%

\newcommand{\pwdhash}{\texttt{PwdHash}\xspace}
\newcommand{\pwdmul}{\texttt{Password Multiplier}\xspace}
\newcommand{\passpet}{\texttt{Passpet}\xspace}
\newcommand{\ours}{\texttt{Dono}\xspace}
\newcommand{\btchr}{1,375,500,956 GH/s}
\newcommand{\btchrsrt}{13.75500956 \times 10^{17}}


\def\sharedaffiliation{
\end{tabular}
\begin{tabular}{c}}

\newenvironment{proofsketch}{\par {\sc Proof Sketch.}}{\hfill\stopproof\par}\def\stopproof{\square}
\def\square{\vbox{\hrule height.2pt\hbox{\vrule width.2pt height5pt \kern5pt
\vrule width.2pt} \hrule height.2pt}}

\pagenumbering{arabic}
\sloppy

\begin{document}

\title{Dono: A Stateless Password Manager}

\numberofauthors{1}
\author{
\alignauthor
Panos Sakkos\\
\email{panos.sakkos@protonmail.com}
}

\maketitle

\begin{abstract}
\hilight{- Draft -}
Passwords are the cornerstone of privacy and security but they are cumbersome to handle.
Password managers offer to take away the pain of maintaining one regularly changing
and strong password per service by creating an account for their user and using it
as the master key of accessing all the passwords. For purposes of usability, they
offer syncing of the data between the different devices that the user uses and for
that purpose, they have to store all the passwords in their backend service. We
propose a purely computational password manager, which requires no data transfer
between devices and doesn't store any password. There is no need for a service that
syncs data across different devices and all the implementation offered is in the
form of native apps. By open-sourcing \ours, the users are able to trust that all
their passwords live at their devices only, while computing the same passwords
across all the different devices used.

\end{abstract}

\section{Introduction}\label{section:introduction}

Ideally a person should handle its passwords by remembering a strong and unique password
per service used and keep updating it regularly with a new unique and strong password.
In practice this is challenging and something that an average person cannot achieve
without investing a lot of time on it. Password managers provide solutions that
offer handling of all the passwords by maintaining a username and a master key of the user.

Local password managers as i.e. Password Safe \cite{passsafe:2015}, offer only client
apps and usually are open-sourced, therefore they can be trusted by the users.
On the other hand, they hurt usability, since there is an inherent need to import
the encrypted passwords after each new installation of the offered clients. As a
result, there are users who are using storage cloud services, like Dropbox i.e., in
order to sync their encrypted password database across their devices. Therefore, the
passwords are can be brute-forced by the companies that offer the storage services, or handed over to a government, in order to be decrypted.

The usability gap of the local password managers is addressed by the online password
managers, like i.e. LastPass, which sync the passwords to their backend services
in order to have them ready to be shipped to any new device that the user logs in
with the same account. This approach cannot be trusted since the user is not in
a position to know the source code that is handling the passwords, even if these
services are open-source. Apart from the lack of trust from the online password
managers, they expand significantly the attack surface, resulting to multiple
vulnerabilities \cite{li2014emperor}.

\ours provides the benefits from both the above approaches by computing, instead of
storing, the passwords. From a user's perspective, the user needs to provide a self-defined $Key$
and a fixed description of the service that needs to retrieve the password for, the $Label$.
For example, the $Labels$ for a LinkedIn and a GitHub account could be $"LinkedIn"$
and $"GitHub"$ respectively. In the case where multiple passwords are needed for a service,
then the $Labels$ can be namespaced. For example the tags for ProtonMail's Login
and Mailbox passwords could be $"ProtonMail.Login"$ and $"ProtonMail.Mailbox"$.
The purpose of the $Label$ is to generate a unique password per service but at the same time
to be easy for the user to remember it. More specifically, the user does not need to remember it,
because this information, based on Section \ref{section:calculations}, can be publicly available.
The $Key$ and the $Labels$ are handled in a way that a new $Key$ will generate
a whole new set of passwords for each $Label$. As a result, when the user wants
 to update all the passwords, then the only information that needs to be updated is the $Key$.

Given that the computation of the passwords is deterministic, there is no need to sync any data
between the user's devices, which means that the computed passwords don't have to
leave the device and reach any backend service. \ours is offered only as open-source
native apps. As a result, the users can trust the \ours apps, by verifying the signatures of the respective produced binary.
A client-only web-app version of \ours is possible, but it's deliberately
avoided in order to mitigate online phishing attacks \cite{chiasson2006usability}.

In the case where a computed password for a service is compromised in plain text by an attacker, 
then the attacker will compromise only this service, given that
the user is using different $Labels$ per service and, based on Section
\ref{section:calculations}, the attacker will not be in a position to compute the $Key$ in a reasonable time.

Summarizing, \ours provides the following:

\begin{enumerate}
  \item Establishing trust with the users about through transparency
  \item Reducing significantly the attack surface of password managers, by replacing their
  storage needs with computations
\end{enumerate}

\section{Related Work}\label{section:related}

\pwdhash \cite{ross2005stronger} was a first attempt to utilize cryptographic hash
functions in order to offer an extra layer of security for passwords.
It was implemented only as a Firefox plugin and was focused on mitigating a
category of phishing attacks which could execute javascript code in order to steal plain text passwords
from the plugin. \pwdhash was generating one password per website but it was expecting existing
passwords as an input, which led to confusion among users \cite{chiasson2006usability}.
Also, the salt was not secret, and it computed based on the website domain, which
also proved to be challenging during the implementation.
The password was not stretched with iterations, but it was suggested as a future step.
Last, web plugins are not anymore a viable solution, given
the rise of the small screen portable devices which usually don't allow third party
plugins to extend the browser capabilities.

\pwdmul \cite{halderman2005convenient} was a Firefox plugin, which was computing
unique passwords per website name, given a master key and the user's username. The
salt used was the username and there were two rounds of hash iterations.
The first round of iterations was computed after the first run of the plugin and
the second when requesting the password for a website. The first round of iterations
was cached, and in the case of its compromise by the attacker, then the computing
cost per password for the attacker was reduced dramatically. Apart from the non-ideal
salt used, the assumed attacker that was brute-forcing the algorithm was a "modern PC"
which would need around 100 seconds in order to guess a password. Since then, the
trend of cryptocurrencies has driven the cost of ASIC hash computers to very low costs
and as a result anyone with a budget of a few thousand dollars can buy a computational
power in the order of GH/s. Also, the computations were not taking into account

\passpet \cite{yee2006passpet} was the continuation of \pwdmul and was focusing on
extending \pwdmul for better usability. It remained a Firefox plugin and petnames
were introduced, which were labels picked by the user and assigned to websites. \passpet
introduced a remote server, which was storing the information of the number of
the first round of hashing iterations and the user's labels.

\section{Password Computation}\label{section:computation}

The $Label$, $l$, is converted to lower case and leading and trailing spaces are trimmed, in order to not force the user remembering the case-sesitivity of the used $Labels$ and to avoid accidental spaces added by phone and tablet keyboards.

A unique per combination of ($Key$,$Label$) salt, $s$, is computed by hashing user's
Key, $k$, appended by the $Label$, $l$, and a fixed $magic salt$ in order to
mitigate encrypted dictionaries in advance \cite{morris1979password}. In our case
the $magic salt$ is equal to "4a5e1e4baab89f3a32518a88c31bc87f618f76673e2cc77ab2127
b7afdeda33b".

$k$, $l$ and $s$ are appended and hashed in order to derive the password, $d$, for $l$.

In order to slow down the attacker \cite{kelsey1998secure}, $d$ is hashed with a salt multiple times
before deriving the final $d$. The number of iterations is determined upon the size of $k$,
in order to protect users from picking a short $k$. For example, a $k$ with length 15, will need
so many iterations that it will not derive the final $d$ in a reasonable time, on a computing device.

\begin{algorithm}[H]
  \caption{Password computation}
  \label{onepasswords}
  \begin{algorithmic}[1] % The number tells where the line numbering should start
    \Procedure{ComputePassword}{$k, l$}
      \Comment{The $Key$ and the $Label$}
      \State $l\gets l.LowerCase().TrimStart(`` ``).TrimEnd(`` ``)$
      \State $s\gets "4a5e1e4b...76673e2cc77ab2127b7afdeda33b"$
      \State $s\gets SHA256(k + l + s)$
      \State $d\gets SHA256(k + l + s)$
      \State $rs\gets DecideRounds(k)$
      \For{$i = 0, i < rs, i\gets i + 1$}

        $s\gets SHA256(d + s)$

        $d\gets SHA256(d + s)$
      \EndFor
      \State \textbf{return} $d$
    \EndProcedure
    \end{algorithmic}
\end{algorithm}

In order to define the number of derivation rounds, $rs$, we need to define an attacker
and the computational time that we want to enforce for brute-forcing all
the possible values of $k$. The function $DecideRounds$ is described in Section \ref{section:calculations}.

Approaches like bcrypt \cite{provos1999bcrypt} for Algorithm \ref{onepasswords} are
not suitable because of the client-side purpose of use.

\section{Rounds of Derivation}
\label{section:calculations}

\subsection{Attacker}
\label{section:calculations:attacker}

In order to define the computation of $rs$ for function $DecideRounds$, we need to
define the attacker that we are protecting against and the computational time that
we want for the brute-forcing of every value of $k$, given a fixed $k$ length, $l$.

Following Kerckhoffs's principle \cite{kerckhoffs1978cryptographie} we assume that
the attacker has knowledge of everything except the $Key$:

\begin{enumerate}
  \item Knows that the victim used Algorithm \ref{onepasswords} for computing passwords
  \item Knows all the victim's computed passwords
  \item Knows all the victim's $Labels$ that were used for computing the passwords
  \item Knows the character set that was used for the victim's $Key$ and its length, $l$
\end{enumerate}

We have assumed that the length of the victim's key is known in order to simplify the calculations. 

Moreover, we assume that the attacker is capable of the following computing capabilities:

\begin{enumerate}
\item Has infinite storage
\item Has compromised all the nodes of the Bitcoin \cite{nakamoto2008bitcoin} network and uses them to compute SHA256 hashes
\end{enumerate}

We have assumed that the attacker has compromised the Bitcoin network, because it's the most powerful hash computer that is known.

\subsection{Victim}
\label{section:calculations:victim}

We assume the following for the victim of the attacker described in \ref{section:calculations:attacker}

\begin{enumerate}
  \item Victim's $Key$ consists only of Latin letters, all in the same case-sensitivity (i.e. only non-capital letters)
  \item Victim's $Labels$ remain the same forever
  \item The victim's $Key$ length, $l$, remains the same forever
\end{enumerate}

\subsection{The attack}
\label{section:calculations:attack}

\subsubsection{Goal}
\label{section:attack:goal}

The goal of the attacker is to find the $Key$ of the victim.

\subsubsection{Computational power}
\label{section:attack:power}

The current global maximum of Bitcoin network's Hash Rate (as of 28th of March 2016) is \btchr \cite{btc:hashrate:2015},
or $\btchrsrt$ hash computations per second and this is the computing power the attacker has available.
The cost, in terms of hashes, of computing one password computed by Algorithm \ref{onepasswords} is $2 + 2 \times rs$ hashes.
The \textit{Password Rate}, $pr$, computed passwords per second, of the attacker
for Algorithm \ref{onepasswords} is:

$$\textit{pr} = \frac{\btchrsrt}{2 + 2 \times rs}$$

\subsubsection{Computational time}

The attack is performed by brute-forcing $k$ in Algorithm \ref{onepasswords}, since all
$l$ and final $d$ values are known.

For a given $k$ length, $l$, the possible $Keys$ of the victim are $26^{l}$.
The time, in (365 days long) years, that the attacker needs to compute all the
passwords for a length $l$, $T_{l}$, is:

$$ \textit{$T_{l}$} = \frac{26^{l}}{31536000 \times \textit{pr}}$$

By upper bounding $T_{l}$ to 1000 years, for each $l$ we can compute $rs(l)$:

$$rs(l) = \lceil \frac{500 \times \btchrsrt \times 31536000 - 26^{l} }{26^{l}} \rceil $$

By encapsulating the calculation of $rs(l)$ in the $DecideRounds$ function, we complete
Algorithm \ref{onepasswords} with Algorithm \ref{deciderounds}.

\begin{algorithm}[H]
  \caption{Determine rounds of derivation}
  \label{deciderounds}
  \begin{algorithmic}[1]
    \Procedure{DecideRounds}{$k$}
    \Comment{The $Key$}
    \State $l\gets k.length()$
    \State $t\gets 1000$
    \Comment{Target brute force years}
    \State $c\gets \btchrsrt$
    \Comment{Computational power}
    \State $rs\gets \lceil \frac{\frac{t}{2} \times c \times 31536000 - 26^{l} }{26^{l}} \rceil$
    \State \textbf{return} $rs$
    \EndProcedure
    \end{algorithmic}
\end{algorithm}

A periodic update of the target computational power, $c$, of the $Password Rate$
is encouraged, in order to keep up with the hardware improvements. This update
will need to provide a migration path to the users, because the change in the
computation will lead to different results, for each pair of $(k, l)$.

Table \ref{table:rounds} lists the different rounds of derivation per $Key$ length.
$Keys$ starting from a length of 17 can be computed in a responsive time even on mobile clients.

\begin{table}[H]
\label{table:rounds}
\centering
\caption{Rounds of Derivation for different $Key$ lengths}
\begin{tabular}{ |l|l| }
  \hline
  $l$ & $rs$ \\
  \hline
  0  & 21688899074207999999999999999\\
  1  & 834188425931076923076923075\\
  2  & 32084170228118343195266271\\
  3  & 1234006547235320892125624\\
  4  & 47461790278281572774061\\
  5  & 1825453472241598952847\\
  6  & 70209748932369190493\\
  7  & 2700374958937276556\\
  8  & 103860575343741405\\
  9  & 3994637513220822\\
  10 & 153639904354646\\
  11 & 5909227090562\\
  12 & 227277965020\\
  13 & 8741460192\\
  14 & 336210006\\
  15 & 12931153\\
  16 & 497351\\
  17 & 19127\\
  18 & 734\\
  19 & 27\\
  20 & 0\\
  \hline
\end{tabular}
\end{table}

The attacker, given the infinite storage capabilities, can store the computed
values of $ComputePassword(k, l)$ for each $k$ that is brute-forced, in order to
compromise service with $Label$ $l$, for every $k$ in the future.
This is mitigated by changing the $magic salt$ of Algorithm \ref{onepasswords} periodically,
i.e. while updating the computational power, $c$, of the $Password Rate$.

\subsubsection{Cost of the attack}

We assume that the attacker is not charged for electricity costs and there are no hardware
failures while the attack is taking place. The arguably best ASIC hash computer
with respect to GH/s per \$ is the \texttt{Spondoolies Tech SP20 Jackson},
which costs \$500 and its advertised capacity is 1,500 GH/s, resulting to \$0.33 per GH/s.
For the computational power of the assumed attacker, this would cost \$45,850,031.
For a high value target, the attacker could invest more money in the attack, in order
to find the victim's $Key$ in a reasonable time, i.e. within a year.
In this case that would cost \$45,850,031,000.

An advanced user that believes that is a high value target, can download the source code,
update the target brute force years and the computational power and then compile
and deploy privately the new hardened \ours binaries. Modifications like this will require a longer
$Key$ in order for the password computation to finish in a responsive time.

\section{Implementation}\label{section:implementation}

\section{Conclusions}\label{section:conclusions}

\bibliographystyle{abbrv}
\bibliography{dono}

\end{document}
